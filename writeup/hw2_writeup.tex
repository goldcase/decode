%
% File naaclhlt2015.tex
%

\documentclass[11pt,letterpaper]{article}
\usepackage{naaclhlt2015}
\usepackage{times}
\usepackage{latexsym}
\usepackage{amsmath}
\usepackage{algorithm}
\usepackage[noend]{algpseudocode}
\setlength\titlebox{6.5cm}    % Expanding the titlebox

\title{HW2 Writeup: Decoding}

\author{Anjali Narayan-Chen \and Johnny Chang\\
{\tt nrynchn2@illinois.edu}, {\tt jychang3@illinois.edu}}

\date{March 10, 2016}

\begin{document}
\nocite{*}
\maketitle

\section{Introduction}

For this homework, we adapt the approach of~\cite{langlais2007greedy} of a local 
greedy search algorithm. Using a predefined set of operations, the greedy search 
proceeds by modifying the current-best translation returned by the default stack decoder
and searching over the set of modified translations in a hill-climbing approach. If
a better-scoring translation is found, the modified translation is returned as the system's
prediction.

\section{Algorithm}
Given $source$, a sentence to translate, the core of the greedy search algorithm proceeds as follows:
\begin{algorithm}
\caption{Greedy Search}
    \begin{algorithmic}[1]
    \State $current \leftarrow {\tt seed}(source)$
    \Loop
        \State $s\_current \leftarrow {\tt score}(current)$
        \State $s \leftarrow s\_current$
        \ForAll{$h\in{\tt neighborhood}(current)$}
            \State $c \leftarrow {\tt score}(h)$
            \If{$c>s$}
                \State $s \leftarrow c$
                \State $best \leftarrow h$
            \EndIf
        \EndFor
        \If{$s = s\_current$}
            \Return $current$
        \Else
            \State $current \leftarrow best$
        \EndIf
    \EndLoop
    \end{algorithmic}
\end{algorithm}

The idea behind the greedy search is to modify parts of the translation one at a time, exploring
the search space of neighboring hypotheses in a hill-climbing fashion. The hope is that such
modifications to the sentence output by the original stack decoder will provide search directions
that lead to improved hypotheses.

In our implementation, the function {\tt seed} that seeds the search with an initial state
is the provided stack decoder. The scoring function, {\tt score}, is similarly defined as the
scoring function used in the provided code, $\log p(\mathbf{f,a|e})+\log p(\mathbf{e})$. We next
define the {\tt neighborhood} function, which takes a candidate translation as an argument and 
returns a set of neighboring hypotheses to consider. 

\subsection{Neighborhood Function}
The neighborhood function is defined via s set of six operations that can transform a current translation,
as defined by~\cite{langlais2007greedy}.

\paragraph{Swap} The \textit{swap} operation swaps two adjacent target segments. This is designed to combat
the baseline model's strong bias toward monotonous translations.

\paragraph{Replace} Given a specific source segment, the \textit{replace} operator exchanges the translation
for that segment with anther found in the phrase translation table.

\paragraph{Bi-Replace} The \textit{bi-replace} operation works similarly to \textit{replace}, allowing the
translation of two adjacent source phrases to change simultaneously. The motivation for this is to modify 
the sentence enough to escape a possible local maximum in the search.

\paragraph{Split} The \textit{split} operation splits a given source phrase into two parts and re-translates
the split phrases according to translations found in the phrase translation table.

\paragraph{Merge} The \textit{merge} operation is the opposite of the \textit{split} operation: it allows two
adjacent source phrases to be merged. The merged phrase then receives a new translation according to the phrase
translation table.


\section{Results}


\bibliographystyle{naaclhlt2015}
\bibliography{hw2_writeup}

\end{document}
